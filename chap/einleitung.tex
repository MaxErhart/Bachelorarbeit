Der Großteil der Masse im Universum besteht aus nicht sichtbarer DM (Dunkler Materie)
deren Existenz durch kosmologische Beobachtungen begründet ist.
Die Zusammensetzung DM ist bis heute unklar. 
Das Untersuchen von DM gibt uns nicht nur Aufschluss über ihre Eigenschaften wie Zusammensetzung, Wechselwirkung und Herkunft, sondern ermöglicht gleichzeitig Erkenntnisgewinn über die Entstehung des Universums.
Prinzipiell liegt ein Fokus jüngerer Experimente auf dem Nachweis einer Streuung von DM mit sichtbarer Materie. 

Viel Aufwand wurde in die Detektion von WIMPs (\textit{weakly interacting massiv particles}) als Kandidaten für DM im Massenbereich von wenigen GeV bis TeV gesteckt.
Zu diesen Experimenten zählen zum Beispiel XENON\cite{Aprile2017} und LUX\cite{DaSilva2017} welche flüssige Edelgase als Detektormaterial verwenden.
Alternativ werden hochreine Halbleiterkristalle bei kryogenen Temperaturen verwendet.
Experimente wie EDELWEISS\cite{EDWIII} oder SuperCDMS\cite{Agnese2018} verwenden Germanium Kristalle.
Um hohe Sensitivitäten zu erreichen werden die Detektoren auf Temperaturen von wenigen Kelvin gekühlt.
Die bei einer Wechselwirkung im Germaniumkristall deponierte Energie in Form von Ionisation, Szintilationslicht oder Wärme (Phononen) ist messbar und gibt Aufschluss über die wechselwirkenden Teilchen. 
Die genaue Energiebestimmung bei EDELWEISS erfolgt durch das Auslesen des Ionisationssignals sowie des Phononsignals.
Die Verwendung beider Kanäle ermöglicht zusätzlich zwischen Kern- und Elektron Streuung zu diskriminieren.
Auf diese Art können große Teile des Parameterbereichs von WIMPs abgedeckt werden.
Bisher wurde allerdings keine eindeutiges WIMP Signal festgestellt werden.
Neben WIMPs ist LDM (\textit{light dark matter}) mit Massen im sub-GeV Bereich eine vielversprechende Möglichkeit.
Das DELight Experiment hat das Ziel mittels DM-Elektron Streuung die Sensitivität im Bereich LDM um mehrere Größenordnungen zu verbessern.
Dazu soll mittels Luke-Verstärkung eine Energieauflösung des Ionisationssignals im $\SI{}{\electronvolt}$ Bereich erreicht werden.

Das für die Luke-Verstärkung nötige Potential soll über eine Vakuumseparierte Elektrode angelegt werden.
Die Aufgabe in dieser Arbeit ist es zu untersuchen, ob es möglich ist mit dem Design der Elektrode ein Ionisationssignal zu messen.
Dazu ist es notwendig die Ausleseelektronik des Ionisationskanal zu entwickeln und deren Rauschen zu untersuchen. 
Die neue Messanordnung soll dann in zukünftigen Arbeiten angewendet werden, um die Linearität des Neganov-Luke-Effekt in einem großen Spannungsbereich zu prüfen.
