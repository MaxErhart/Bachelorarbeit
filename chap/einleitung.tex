Der Großteil der Masse im Universum besteht aus nicht sichtbarer \ac{DM} deren Existenz durch kosmologische Beobachtungen begründet ist.
Die Zusammensetzung von \acs{DM} ist bis heute unklar. 
Das Untersuchen von \acs{DM} gibt uns nicht nur Aufschluss über ihre Eigenschaften wie Zusammensetzung, Wechselwirkung und Herkunft, sondern ermöglicht gleichzeitig Erkenntnisgewinn über die Entstehung des Universums.
Prinzipiell liegt ein Fokus jüngerer Experimente auf dem Nachweis einer Streuung von \ac{DM} mit sichtbarer Materie. 

Viel Aufwand wurde in die Detektion von \acp{WIMP} als Kandidaten für DM im Massenbereich von wenigen GeV bis TeV gesteckt.
Auf Direktem Wege wird anhand \ac{DM}-Nukleon Streuung nach \acp{WIMP} gesucht.
Zu diesen Experimenten zählen zum Beispiel \acs{XENON}\cite{Aprile2017} und \acs{LUX}\cite{DaSilva2017} welche flüssige Edelgase als Detektormaterial verwenden.
Alternativ werden hochreine Halbleiterkristalle bei kryogenen Temperaturen verwendet.
Experimente wie \acs{EDELWEISS}\cite{EDWIII} oder \acs{SuperCDMS}\cite{Agnese2018} verwenden Germanium Kristalle in einem elektrischen Feld.
Um hohe Sensitivität zu erreichen werden die Detektoren auf Temperaturen von wenigen Kelvin gekühlt.
Die bei einer Wechselwirkung im Germaniumkristall deponierte Energie in Form von Ionisation, Szintilationslicht oder Wärme (Phononen) ist messbar und gibt Aufschluss über die wechselwirkenden Teilchen. 
Die Informationen über das Ereignis werden bei EDELWEISS durch das Auslesen des Ionisationssignals sowie des Phononsignals gewonnen.
Die Verwendung beider Kanäle ermöglicht zusätzlich zur Energiebestimmung zwischen Kern- und Elektron Streuung zu diskriminieren.
Auf diese Art können große Teile des Parameterbereichs von \acp{WIMP} abgedeckt werden.
Bisher konnte allerdings kein eindeutiges \ac{WIMP} Signal festgestellt werden.
Neben \acp{WIMP} ist \ac{LDM} mit Massen im sub-GeV Bereich eine vielversprechende Möglichkeit.
Das DELight Experiment hat das Ziel mittels \ac{DM}-Elektron Streuung die Sensitivität im Bereich \ac{LDM} um mehrere Größenordnungen zu verbessern.
Dazu soll mittels Luke-Verstärkung (siehe Abschnitt \ref{sec:LukeAmp}) eine Energieauflösung des Ionisationssignals im $\SI{}{\electronvolt}$ Bereich erreicht werden.

Der Aufbau sieht für das Einbringen des elektrischen Feldes vakuumseparierte Elektroden vor.
Dies verbessert einerseits die Qualität des Wärmekanals erhöht aber andererseits die Anforderungen für die Messung im Ionisationskanal.
Die Aufgabe in dieser Arbeit ist es zu untersuchen, ob es möglich ist an den vakuumseparierten Elektroden ein Signal im Ionisationskanal zu messen.
Dazu ist es notwendig die Verstärkerelektronik des Ionisationskanal zu entwickeln und deren Auflösung zu untersuchen.
Ziel ist es die Messanordnung in zukünftigen Arbeiten anzuwenden, um die Linearität des Neganov-Luke-Effekts in einem großen Spannungsbereich zu prüfen.


