Der Nachweis DM durch Streuung an einem SM Teilchen wird als direkter Nachweis bezeichnet.
Dabei wird im Experiment die bei der Streuung deponierte Energie in Form von Ionisation, Szintillationslicht oder Phononen bestimmt.
Die Rate solcher Ereignisse ist entscheidend von der Dichte, relativen Geschwindigkeit zwischen Erde und DM-Halo, Masse der DM Teilchen und Wirkungsquerschnitt der Wechselwirkung abhängig.
Aufgrund des kleinen Wirkungsquerschnitt ist genaue Kenntnis und Minimierung des Untergrunds notwendig.
Daher befinden sich Experimente dieser Art in Laboren tief unter der Erde.
Aufgrund natürlicher Radioaktivität wird der Untergrund zusätzlich durch aktive und passive Schilde sowie hochreines Detektormaterial verringert.

Im Wesentlichen gibt es zwei Arten von Detektor Typen kryogene Halbleiterdetektoren und Edelgasdetektoren mit flüssigem Edelgas.
Flüssig Edelgasdetektoren verwenden Photomultiplier um das Scintillationslicht welches bei Wechselwirkungen entsteht zu detektieren.
Zusätzlich driften die Ladungsträger des Ionisationssignal im extern angelegten Feld und erzeugen dabei weiteres Szintillationslicht.
Dadurch kann zwischen Nukleon und Elektron Streuung unterschieden werden.
Der Detektor fungiert dadurch als Time Projection Chamber.
Aktuell wird als Detektormaterial flüssiges Xenon oder flüssiges Argon verwendet.
Experimente dieser Art sind XENON\cite{Aprile2017} und LUX\cite{DaSilva2017}.
Kryogene Halbleiterdetektoren sind hochreine Kristalle welche im $\SI{}{\milli\kelvin}$ Bereich angewendet werden.
Über Sensoren an der Oberfläche wird anhand des Wärme- und Ionisationssignal die deponierte Energie bestimmt.
Prominente Beispiele für Experimente dieser Art sind EDELWEISS\cite{EDWIII} und SuperCDMS\cite{Agnese2018}.