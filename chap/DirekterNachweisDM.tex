Der Nachweis DM durch Streuung an einem SM Teilchen wird als direkter Nachweis bezeichnet.
Dabei wird im Experiment die bei der Streuung deponierten Energie in Form von Ionisation, Szintillationslicht oder Phononen bestimmt.
Die Rate solche Ereignisse ist entscheidend von der Dichte, relativen Geschwindigkeit zwischen Erde und DM Halo, Masse der DM Teilchen und Wirkungsquerschnitt der Wechselwirkung abhängig.
Aufgrund des kleinen Wirkungsquerschnitt ist genaue Kenntnis und Minimierung des Untergrunds notwendig.
Daher befinden sich Experimente dieser Art in Laboren tief unter der Erden.
Aufgrund natürlicher Radioaktivität wird der Untergrund zusätzlich durch aktive und passive Schilde sowie hochreines Detektormaterial verringert.

Im Wesentlichen gibt es zwei Arten von Detektor Typen, kryogene Halbleiterdetektoren und flüssig Edelgasdetektoren.
Kryogene Halbleiterdetektoren sind hochreine Kristalle welche im $\SI{}{\milli\kelvin}$ Bereich operiert werden.
Über Sensoren wird anhand des Wärme- und Ionisationssignal die deponierte Energie bestimmt.
Prominente Beispiele für Experimente dieser Art sind EDELWEISS\cite{EDWIII} und SuperCDMS\cite{Agnese2018}.
Flüssig Edelgasdetektoren bestehen aus großen Tanks gefüllt mit flüssigem Edelgas.
Aktuell werden flüssig Xenon und flüssig Argon Detektoren verwendet.
Über Sensoren an der Oberfläche wird das Scintillationslicht und Ionisationssignal bestimmt.
Eine Auswahl Experimente dieser Art sind XENON\cite{Aprile2017} und LUX\cite{DaSilva2017}