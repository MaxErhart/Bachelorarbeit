Aus dem Einfluss DM aufgrund von gravitativen Effekten lässt sich nur ein unvollständiges Bild ihrer Zusammensetzung ableiten.
Daher wird auf viele verschiedene Arten versucht Aufschluss über ihre Eigenschaften zu gewinnen.
Die dabei verwendeten Methoden können in drei Klassen entsprechend 

\subsection*{Indirekter Nachweis}

Messungen von Teilchen des Standardmodells, welche bei der Annihilation oder dem Zerfall von Dunkler Materie entstehen, wird als Indirekter Nachweis bezeichnet. 
Die Rate solcher Ereignisse ist in Gebieten hoher DM Dichte am größten.
Eine große Schwierigkeit bei der indirekten Suche sind die zahlreichen unterschiedlichen Quellen von Teilchen, welche eine ähnliche Signatur aufweisen.

Experimente wie H.E.S.S. oder VERITAS nutzen Cherenkov Teleskope um in Zentren von Galaxien nach Dunkler Materie zu suchen.
Das Weltraumteleskop PAMELA berichtete 2009 von einem 9 bis 275$\si{\giga\electronvolt}$ Positron zu Elektron Überschuss.
Das AMS-02 Experiment auf der \textit{International Space Station} konnte diese Ergebnisse bestätigen.
Um aus diesen Ergebnisse eindeutig auf die Annihilation von Dunkler Materie zu schließen ist eine genauso großer Überschuss von Protonen zu Antiprotonen notwendig.
Dieser konnte bisher nicht beobachtet werden.\cite{PAMELA}