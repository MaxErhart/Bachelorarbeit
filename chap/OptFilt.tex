In diesem Abschnitt werden ein paar grundlegende Erkenntnisse der optimal filtering Methode dargestellt.
Die Darstellung orientiert sich anhand \cite{Golwala2000, Enss2005}.

\subsection{Diskrete Fouriertransformation}
In der Realität können Signale nicht kontinuierlich Abgetastet werden und nur über einen begrenzten Zeitraum aufgenommen.
Die Fouriertransformation muss daher für den diskreten Fall angepasst werden zu
\begin{align*}
\widetilde{v}_n &= \frac{1}{N}\sum_{k=-N/2}^{N/2} v_k e^{-i2\pi f_n t_k} \\
v_k &= \sum_n \widetilde{v}_{n=n=-N/2}^{N/2} e^{i2\pi f_n t_k}
\end{align*}
mit der Spurlänge $N=Tf_s$ welche sich aus der Spurdauer $T$ und sample Rate $f_s$ ergibt, Zeitpunkten $t_k = k\Delta T = k/f_s$ und Frequenzen $f_n = n/T$.
Zu sehen ist, dass es zu Frequenz bins der Breite $f_{n+1}-f_n=1/T$ kommt aufgrund der endlichen Spurdauer das gleich Prinzip tritt bei der Unschärferelation auf.
Außerdem ist die Bandbreite durch die Nyquest-Frequenz $f_{Nq}=f_{N/2}=N/2T=f_s/2$, der halben Abtastrate, nach oben und unten begrenzt.

\subsection{Rauschen}
Die Fluktuationen der Spannung werden als Gaussverteilt angenommen mit der Varianz $\langle\left[v(t)\right]^2\rangle$.
Die Varianz beschreibt das Rauschen allerdings nicht vollständig da Korrelationen des Signals nicht berücksichtigt werden.
Korrelationen treten auf da eine Fluktuation der Spannung mit einer bestimmte Zeitkonstante $\tau$ abfällt und daher $v(t)$ Informationen über $v(t+\tau)$ enthält.
Die Autokorrelationsfunktion $R(\tau)$ enthält diese
\begin{align*}
R(\tau) &= \langle v(t)v(t+\tau)\rangle \\
&= \lim_{T\rightarrow \infty}\frac{1}{T}[v\otimes v](\tau) \\
&= \lim_{T\rightarrow \infty}\frac{1}{T}\int_{-T/2}^{T/2}\,dt\, v(t)v(t+\tau)
\end{align*}
$\otimes$ steht hier für die Kreuzkorrelation.
Die Informationen sind auch im Frequenzspektrum des Rauschen enthalten mit dem Vorteil, dass für lineare Systeme das Rauschen unterschiedlicher Frequenzen unkorreliert ist.
Die spektrale Leistungsdichte $J(f)$ ist gegeben durch die Fouriertransformation der Autokorrelationsfunktion und hat die Einheit $\SI{}{\volt^2\per\hertz}$
\begin{equation}
J(f) = \lim_{T\rightarrow \infty}\int_{-T}^{T}\, dt\, R(t) e^{-jwt}.
\end{equation}
Entsprechend gilt 
\begin{align*}
R(t) &= \lim_{T\rightarrow \infty}\int_{-T}^{T}\,df\, J(f) e^{jwt} \\
\Rightarrow \langle\left[v(t)\right]^2\rangle &= R(0) = \int_{-\infty}^{\infty}\,df\, J(f).
\end{align*}
Das Integral der spektralen Leistungsdichte gibt also die Varianz des Rauschen. 

Die spektrale Leistungsdichte wird in der Regel nicht aus der Autokorrelationsfunktion bestimmt sondern direkt aus Fourier transformierten Spuren ohne Signale.
Es gilt
\begin{align*}
J(f) &= \lim_{T\rightarrow \infty}\int_{-T}^{T}\,dt\, R(t) e^{-jwt} \\
&= \lim_{T\rightarrow \infty}\int_{-T}^{T}\,dt\,e^{-jwt}\lim_{T\rightarrow \infty}\frac{1}{T}[\otimes](t) \\
&= \lim_{T\rightarrow \infty}\int_{-T}^{T}\,dt\,e^{-jwt}\lim_{T\rightarrow \infty}\frac{1}{T} \int_{-\infty}^\infty\,df_1\,e^{iw_1t}\widetilde{v}^*(f_1)\widetilde{v}(f_1)\\
&= \lim_{T\rightarrow \infty}\frac{1}{T}\int_{-\infty}^\infty\,df_1\,e^{iw_1t}|\widetilde{v}(f_1|^2\delta(f-f_1) \\
&= \lim_{T\rightarrow \infty}\frac{1}{T} |\widetilde{v}(f)|^2 
\end{align*}
hierbei wurde im dritten Schritt ausgenutzt, dass für die Fouriertransformation der Kreuzkorrelation gilt
\begin{equation}
[g\otimes h](t) \stackrel{\mathcal{FT}}{=} \widetilde{g}^*(f)\widetilde{h}(f).
\end{equation}
Für den Fall diskreter Signale wird die Ersetzung $\widetilde{v}(f)\rightarrow T\widetilde{v}_n$ gemacht und der Grenzwert fallen gelassen
\begin{equation}
J(f_n) = \frac{N}{f_s}|\widetilde{v}_n|^2.
\end{equation}
Dies ist die übliche Form um $J(f)$ zu bestimmen.
Mehrere Spuren ohne Signal werden aufgenommen, $|\widetilde{v}_n|^2$ mittels DFT bestimmt und ihr Mittelwert bestimmt.
Werden nur die positiven Frequenzen betrachtet muss die doppelseitige spektrale Leistungsdichte $J(f)$ um einen Faktor zwei korrigiert werden
\begin{equation}
J_{ss}(f) = 2J(f) \quad f > 0.
\end{equation}

\subsection{Optimaler Pulshöhen Fit}
Ein realer Puls hat die Form
\begin{equation}
v(t) = As(t) + n(t)
\end{equation}
mit einer Rauschspur $n(t)$ und der erwarteten Pulsform $s(t)$ mit Amplitude $A$.
Die spektrale Leistungsdichte sei gegeben durch $J(f)$.
Um die beste Amplitude zu bestimmen wird ein $\chi^2$-Fit der erwarteten Pulsform an den realen Puls durchgeführt.
Der Fit wird im Frequenzraum durchgeführt da Unterschiedliche Frequenzanteile unkorreliert sind
\begin{equation}
\chi^2 = \int_{-\infty}^\infty\, df \frac{|\widetilde{v}(f) - A\widetilde{s}(f)|^2}{J(f)}.
\end{equation}
Durch die Minimierung von $\chi^2$ erhält man für den besten Schätzer
\begin{equation}
\hat{A} = \frac{\int_{-\infty}^\infty\, df \frac{\widetilde{v}(f)\widetilde{s}^*(f)}{J(f)}}{\int_{-\infty}^\infty\, df \frac{|\widetilde{s}(f)|^2}{J(f)}}.
\end{equation}
Für die Varianz auf den Schätzer ergibt sich
\begin{equation}
\sigma^2_A = \left[\frac{1}{2}\frac{\partial^2}{\partial A^2}\chi^2\right]^{-1} = \left[\int_{-\infty}^\infty\, df \frac{|\widetilde{s}(f)|^2}{J(f)} \right ]^{-1}.
\end{equation}
Die Varianz auf die Amplitude bestimmt die beste erreichbare Auflösung bei der gegebenen Pulsform.
Für den Übergang zum diskreten Fall werden die Ersetzungen 
\begin{align*}
\widetilde{s}^*(f) &\rightarrow \frac{N}{f_s} \widetilde{s}^*_n \\
\widetilde{s}(f) &\rightarrow \frac{N}{f_s} \widetilde{s}_n \\
J(f) &\rightarrow J(f_n) \\
\int_{-\infty}^\infty\, df &\rightarrow \frac{f_s}{N}\sum_{n=-N/2}^{N/2}
\end{align*}
und führen zu
\begin{equation}
\sigma^2_A = \left[\frac{N}{f_s} \sum_{n=-N/2}^{N/2} \frac{|\widetilde{s}_n|^2}{J(f_n)}\right]^{-1}.
\end{equation}
