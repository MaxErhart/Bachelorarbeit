MMCs bestehen aus einem Absorber welcher thermisch stark an einen paramagnetischen Temperatursensor gekoppelt ist.
Der Sensor ist wiederum schwach an ein thermisches Bad gekoppelt.
Das Volumen des Sensor ist mit einem schwachen Magnetfeld durchsetzt und führt zu einer Magnetisierung entsprechend dem Curie-Gesetz $M \propto T^{-1}$.
Eine Temperaturerhöhung aufgrund der deponierten Energie $\delta E$ führt zu einer Änderung der Magnetisierung
\begin{equation}
\delta M = \dfrac{M}{T}\frac{\delta E}{C_{tot}}.
\end{equation}
Die Änderung der Magnetisierung wird in Form einer Änderung des magnetischen Flusses durch eine supraleitende picuip coil ausgelesen.
Diese Spule erzeugt gleichzeitig das notwendige Magnetfeld.
Die Änderung des magnetischen Flusses in der Spule wird auf einen \ac{SQUID} übertragen welcher diesen in ein entsprechendes Spannungssignal umwandelt.

Ein Schwachpunkt von MMCs ist die lange Zeit ($\sim\SI{}{\milli\second}$) bis sich ein Gleichgewicht zwischen dem Phonon- und dem Spin-System einstellt aufgrund ihres geringen Energieaustausch bei Temperaturen im $\SI{}{\milli\kelvin}$ Bereich.
Um dies zu umgehen wird ein mit magnetischen Ionen dotiertes Metall verwendet.
Dies hat den Vorteil, dass die starke Kopplung der Elektronen im Leitungsband mit dem Spin-System zu einer schnellen Thermalisierung führt.
Der Nachteil ist eine größere Wärmekapazität und eine geringere Temperaturabhängigkeit der Magnetisierung aufgrund von \ac{RKKY} Wechselwirkungen.
Ein häufig verwendetes Material ist AuEr.



