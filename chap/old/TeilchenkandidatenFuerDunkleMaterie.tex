Die wenigen bisher Diskutierten Experimenten legen bereits einige für Dunkle Materie notwendige Eigenschaften fest.
Sie muss ungeladen sein, da sie nicht elektromagnetisch wechselwirkt.
Auch muss ihre Wechselwirkung mit Sichtbarer Materie und ihre Selbstwechselwirkung extrem selten sein.
Außerdem ist ihre Lebensdauer extrem Lang da sie sonst bereits zerfallen wäre.

Es gibt eine große Anzahl verschiedener Kandidaten für Dunkle Materie.
Dabei muss zwischen heißer Dunkler Materie und kalter Dunkler Materie unterschieden werden.
Heiße Dunkle Materie bewegt sich mit relativistischen Geschwindigkeiten.
Der bedeutendste Kandidat für Heiße Dunkle Materie ist ein leichtes Neutrino.
Kalte Dunkle Materie bewegt sich hingegen mit nicht relativistischen Geschwindigkeiten.

Als \textit{weakly interacting particle} (WIMP) wird eine Klasse von Kandidat für kalte Dunkle Materie bezeichnet.
Die Masse von WIMPs liegt in einem Bereich von $\SI{10}{\giga\electronvolt}$ bis einige $\SI{}{\tera\electronvolt}$.
Die vielversprechendste Kandidaten für WIMPs sind leichte Supersymmetrische Teilchen.
Der prominenteste unter ihnen ist das Neutralino.\cite{DMCandidates}

Das Axion ist ein Teilchen welches ursprünglich eingeführt wurde um die CP-Verletzung zu erklären.
Zusätzlich erweisen sich das Axionen hervorragend als Kandidat für Dunkle Materie.
Die Wechselwirkung von Axionen mit gewöhnlicher Materie ist extrem selten.
Die Masse des Axion ist extrem kleine($\sim\SI{}{\mu\electronvolt}$).
Der supersymmetrische Partner des Axion ist auch ein möglicher Kandidat.

Ein Kandidat für warme Dunkle Matrie könnte das sterile Neutino sein.
Dabei handelt es sich um Neutrinos ähnlich zu denen des Standartmodells.
Allerdings sind diese rechtshändig und koppeln somit nicht über die schwache Wechselwirkung mit Teilchen des Standartmodells.
Es wird angenommen, dass die Masse dieser Teilchen in der Größenordnung von $\SI{}{\kilo\electronvolt}$ liegt.\cite{Drewes}