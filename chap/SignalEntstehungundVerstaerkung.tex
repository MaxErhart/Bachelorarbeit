Im Detail zu verstehen wie die im Detektor entstehenden Signale auf eine messbare Form gebracht werden ist von großer Bedeutung.
In kryogenen Halbleiterdetektoren entstehen Ionisationssignale indem mittels Ladungsverstärkern im Detektor driftende Ladungen elektrische Signale erzeugen.
Die Kenntnis über diese Umwandlung erlaubt es aus dem gemessenen Wert Aufschluss über das im Detektor stattgefundene Ereignis zu gewinnen.
Daher möchte ich in diesem Kapitel auf die Erzeugung und Verstärkung der durch Ionisation im Detektor entstehenden elektrischen Signalen eingehen.