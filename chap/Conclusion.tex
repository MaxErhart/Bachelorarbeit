Damit die Luke-Verstärkung überprüft werden kann müssen Ionisationssignale sowohl über den Luke-Effekt im Wärmekanal als auch über den Ramo-Effekt im Ionisationskanal bestimmt werden.

Aufgrund von Oberflächen Ströme welche mit größerer Spannung zunehmen und einem Verlust von Wärme durch die Verbindungskabel der Elektrode soll eine vakuumseparierte Elektrode verwendet werden.
Ein Nachteil der vakuumseparierten Elektrode ist ein abgeschwächtes Signal im Ionisationskanal.

In dieser Arbeit wurde die Verstärkerelektronik entwickelt um den Ionisationskanal auszulesen.
Da die ohnehin kleinen Signale aufgrund der vakuumseparierten Elektrode weiter abgeschwächt werden sollte die Elektronik eine möglichst große Eingangsimpedanz und ein möglichst kleines Rauschen haben.

Für die Elektronik wurde ein auf HEMTs basierter Ansatz gewählt.
HEMTs funktionieren selbst bei kryogenen Temperaturen weshalb die Elektronik in unmittelbarer nähe zum Detektor im Kryostaten installiert werden kann.
Dies führt zu besserer Signalqualität.
Der Nachteil handelsüblicher HEMTs sind allerdings große Leckströme und großes 1/f-Rauschen.

Die Elektronik wurde bei Raumtemperatur und flüssig Stickstofftemperatur getestet.
Bei Raumtemperatur ist der Leckstrom aller HEMTs zu groß um die Elektronik wie vorgesehen mit Vorgespannten Kondensatoren und offenen Relais zu verwenden.
Mit abnehmender Temperatur nimmt allerdings auch der Leckstrom ab, sodass es möglich war einen der verwendeten HEMTs über einen längeren Zeitraum bei offenem Relais einzusetzen.

Mit allen HEMTs wurde eine Spannungsverstärkung in der Größenordnung von $\mathcal{O}(10)$ erreicht.

Das Beste Rauschen wurde bei flüssig Stickstoff Temperatur und offenen Relais aufgenommen und ist in Abb. \ref{fig:54ROpen} dargestellt.
Bei einer Frequenz von \SI{1}{\hertz} liegt es bei \SI{0,36}{\micro\volt\per\sqrt\hertz}.

Für den Leckstrom wurde ein Wert von \SI{9,2}{\femto\ampere} bestimmt.

Aus der spektralen Leistungsdichte des Rauschen wurde mittels der optimal filtering Methode (siehe Abschnitt \ref{sec:OptFilt}) eine Vorhersage der erwarteten Energieauflösung anhand des erwarteten Signalpuls gemacht.
Bei einer Eingangskapazität von \SI{100}{\pico\farad} ergibt sich für die Energieauflösung $\sigma_E = 1/(a-b)\cdot\SI{0,731}{\kilo\electronvolt}$.
Der Prozentsatz des Potentials welches im Detektorvolumen abfällt $(a-b)$ und daher von den Ladungsträgern durchlaufen werden kann nimmt entscheidend Einfluss auf die erreichbare Energieauflösung.

Als nächste Schritte sollte die Elektronik unter den Bedingungen wie sie im Experiment gegeben sind, $\SI{4}{\kelvin}$ und mit einem Richtigen Detektor, getestet werden.
Eine Möglichkeit um das Rauschen und den Leckstrom noch weiter zu optimieren ist es CNRS HEMTs zu verwenden.
Mit diesen ist es allerdings aufgrund ihrer großen Kapazität empfehlenswert eine Kaskode als Verstärker zu verwenden um den Miller-Effekt zu umgehen.