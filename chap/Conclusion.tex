Ein Detektor mit Vakuum separierte Elektrode hat den Vorteil einer besseren Energieauflösung im Wärmekanal auf kosten eines abgeschwächten Signals im Ionisationskanal.
Um die Luke-Verstärkung zu überprüfen ist es notwendig das Ionisationssignal in beiden Kanälen auszulesen.
im Ionisationskanal sollen dazu Signale von $\gamma$-Quellen im $\SI{}{\kilo\electronvolt}$ Bereich sichtbar sein.
In dieser Arbeit wurde eine Verstärkerelektronik entwickelt um den Ionisationskanal auszulesen sie fungiert gleichzeitig als Impedanzwandler damit der Signalstrom nicht von den langen Kabeln und der Digitalisierung belastet wird.
Diese ist dazu ausgelegt ohne zusätzliche Heizung im Kryostaten bei $\SI{4}{\kelvin}$ zu operieren.
Dazu wurden handelsübliche HEMTs verwendet mit dem Vorteil der leichten Verfügbarkeit aber unter großen Leckströmen und großem niederfrequentem Rauschen leiden.
Die Elektronik wurde bei Raumtemperatur und flüssig Stickstofftemperatur getestet.
Bei Raumtemperatur ist der Leckstrom aller HEMTs zu groß um die Elektronik wie vorgesehen mit Vorgespannten Kondensatoren und offenem Relais zu operieren.
Mit abnehmender Temperatur nimmt allerdings auch der Leckstrom ab, sodass es möglich war einen HEMT über einen längeren Zeitraum bei offenem Relais zu operieren.
Mit allen HEMTs wurde eine Spannungsverstärkung in der Größenordnung von $\mathcal{O}(10)$ erreicht.
Das Rauschen ist etwa einen Faktor $10$ größer als das der EDELWEISS-III bei $\SI{4}{\kelvin}$ Elektronik\cite{EDWIII}.
Ein Teil des Rauschen kann durch den siedenden Stickstoff verursacht sein in welchen die Elektronik eingetaucht wurde.
Dafür spricht, dass das Rauschen im Kalten größer wurde.
Die selbe Beobachtung wurde auch schon in der Arbeit von Axel Gullasch\cite{Gullasch2015} gemacht.
Ein Leckstrom von $\SI{1,4}{\femto\ampere}$ wurde bestimmt.
Aus dem Leistungsdichtespektrum des Rauschen wurde mittels der optimal filtering Methode(siehe Abschnitt \ref{sec:OptFilt}) die Energieauflösung anhand des erwarteten Signalpuls zu $\frac{1}{a-b}\SI{1,31}{\kilo\electronvolt}$ bestimmt.
Um die Energieauflösung mit der EDELWEISS-III Elektronik von $\SI{500}{\electronvolt}$ zu vergleichen wurde auch eine $\SI{150}{\pico\farad}$ Detektorkapazität angenommen.
Die Elektronik sollte noch unter den Bedingungen wie sie im Experiment gegeben sind, $\SI{4}{\kelvin}$ und ein Richtiger Detektor, getestet werden.
Um die Schaltung noch weiter zu optimieren besteht die Möglichkeit CNRS/LPN HEMTs zu verwenden welche hervorragende Eigenschaften für unsren Anwendungsfall aufweisen. 