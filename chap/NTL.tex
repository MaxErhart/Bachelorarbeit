Die Fähigkeit Ionisationssignale einzelner Elektronen zu messen ist für die Suche nach leichter Dunkler Materie von großem Interesse.
Das Rauschen in der Ladungsbestimmenden Ausleseelektronik limitiert allerdings die Energieauflößung auf die Größenordnung von $\SI{100}{\electronvolt}$.
Da das Ionisations- und Phononsignal nicht unabhängig ist, ist es möglich die Phononen zu bestimmen welche entstehen wenn Ladungsträger im Kristall driften.
Die Energie der Phononen ist proportional zur angelegten Driftspannung $V_b$ und werden als Luke-Phononen bezeichnet.
Das gesamte Phononsignal setzt sich dann aus dem der initialen Wechselwirkung und der Luke-Phononen zusammen
\begin{equation}
E_P = \frac{E_{dep}}{\epsilon}eV_b + E(1-\frac{\delta}{\epsilon})
\end{equation}
$V_b$ ist die durchlaufene Spannung der Ladungsträger, $e$ die Elementarladung, $\delta$ die minimale Ionisationsenergie und $\epsilon$ die mittlere Energie um eine Elektron-Loch-Paaren zu erzeugen.
Die Konstante $\epsilon$ ist abhängig davon welches Material verwendet wird und ob es sich um Elektron- oder Nukleonstreuung handelt.
Für Elektronstreuung an einem Germaniumkristall ist $\epsilon=\SI{3}{\electronvolt}$\cite{Luke1988}.
Die Anzahl der Elektron-Loch-Paare ist gegeben durch
\begin{equation}
N_{eh} = \frac{E_{dep}}{\epsilon}.
\end{equation}

Der bisherige Ansatz bei EDELWEISS war das initiale Phononsignal nicht zu maskieren indem Biasspannungen in der Größenordnung von $\SI{1}{\volt}$\cite{Arnaud2016} angelegt werden.
Dadurch ist es möglich aus dem gemessenen Phonon- und Ionisationssignal das Phononsignal der ursprünglichen Wechselwirkung zu berechnen.

Alternativ kann durch große Driftspannung die Anzahl von Luke-Phononen pro driftendem Ladungsträger nahezu beliebig groß gewählt werden.
Das rauschen der Ausleseelektronik für das Phononsignal bleibt dabei allerdings unverändert.
Dadurch wird das Signal zu Rausch Verhältnis besser umso höher die Driftspannung ist.
Dabei geht jedoch die Information über das initiale Phononsignal verloren.
Die Absicht ist es auf diese weise einzelne Elektron-Loch-Paare in Germanium auflösen zu können\cite{Mirabolfathi2015}.
