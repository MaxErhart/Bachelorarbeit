Die Fähigkeit Ionisationssignale einzelner Elektronen zu messen, ist für die Suche nach LDM von großem Interesse.
Das Rauschen im Ionisationskanal limitiert allerdings die Energieauflösung.
Da Ionisations- und Phononsignal nicht unabhängig sind, ist es möglich das Ionisationssignal zu bestimmen indem das sekundäre Phononsignal gemessen wird, welches entsteht wenn Ladungsträger im Kristall driften.
Die Energie der sekundären Phononen ist proportional zur angelegten Driftspannung $V_b$ sie werden als Luke-Phononen bezeichnet.
Die gesamte Energie der Phononen $E_P$ setzt sich dann aus dem der initialen Wechselwirkung und der Luke-Phononen zusammen
\begin{equation}
E_P = \frac{E_{dep}}{\epsilon}eV_b + E(1-\frac{\delta}{\epsilon})
\end{equation}
$V_b$ ist die durchlaufene Spannung der Ladungsträger, $e$ die Elementarladung, $\delta$ die minimale Ionisationsenergie und $\epsilon$ die mittlere Energie um ein Elektron-Loch-Paar zu erzeugen.
Die Konstante $\epsilon$ ist abhängig davon, welches Material verwendet wird und ob es sich um Elektron- oder Nukleonstreuung handelt.
Für Elektronstreuung an einem Germaniumkristall ist $\epsilon=\SI{3}{\electronvolt}$\cite{Luke1988}.
Die Anzahl der Elektron-Loch-Paare ist gegeben durch
\begin{equation}
N_{eh} = \frac{E_{dep}}{\epsilon}.
\end{equation}

Ein Ansatz ist das initiale Phononsignal nicht zu maskieren indem Driftspannungen in der Größenordnung von $\SI{1}{\volt}$ angelegt werden.
Dadurch ist es möglich aus dem gemessenen Phonon- und Ionisationssignal das Phononsignal der ursprünglichen Wechselwirkung zu bestimmen.

Alternativ kann durch große Driftspannung die Anzahl von Luke-Phononen pro driftendem Ladungsträger nahezu beliebig groß gewählt werden.
Das Rauschen des Wärmekanals bleibt dabei allerdings unverändert.
Dadurch wird das Signal zu Rausch Verhältnis besser umso höher die Driftspannung ist.
Dabei wird jedoch die Information über das initiale Phononsignal maskiert.
Die Absicht ist es auf diese Weise einzelne Elektron-Loch-Paare in Germanium auflösen zu können\cite{Mirabolfathi2015}.
